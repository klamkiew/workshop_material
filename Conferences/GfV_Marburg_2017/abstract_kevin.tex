\documentclass{article}
\usepackage[utf8]{inputenc}
\usepackage{color}
\usepackage{csquotes}

\begin{document}
\begin{center}
	\textbf{A comprehensive overview of microRNAs in RNA viruses}
\end{center}
\noindent
Introduction/Question\\
In the last years several microRNAs (miRNAs) encoded by viruses were identified.
Hitherto there are 500 viral miRNAs distributed to 29 virus species submitted to the \textit{mirbase.org} database.
However, from these 500 miRNAs, only 18 were discovered in three different RNA virus species.
MiRNAs encoded in RNA viruses were found in the Dengue virus 2 (see Hussain and Asgari, 2014), the West Nile virus (see Hussain \textit{et al.}, 2012), the enterovirus 71 (see Weng \textit{et al}, 2014)
and the hepatitis A virus (see Shi \textit{et al.}, 2016).
To date, a comprehensive overview of miRNAs in RNA viruses is not available.
\\ \ \\
Method\\
We propose a genome-wide, alignment-based overview for \textit{de-novo} precursor miRNAs located in RNA virus genomes.
We combined and modified several established methods for eukaryotic miRNA prediction based on support vector machines, covariance models, stable secondary structures of precursor miRNAs and pattern recognition.
Furthermore, we combined these predictions with RNA-seq data (public available and generated in our lab).
\\ \ \\
Results\\
With this systematic approach, we detected the majority of already known \mbox{miRNAs} and annotated novel miRNA candidates throughout several RNA virus families.
\\ \ \\
Conclusion\\
Emerging data indicates viral miRNAs to either regulate viral replication or suppress counter mechanisms in infected cells.
Previous studies show that miRNA mimics and inhibitors decrease the viral RNA level in infected cells.
Therefore, a comprehensive overview of miRNAs in RNA viruses might lead to a general approach to disturb viral replication and may lead to new  medical therapies.
\end{document}